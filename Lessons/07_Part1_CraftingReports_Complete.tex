% Options for packages loaded elsewhere
\PassOptionsToPackage{unicode}{hyperref}
\PassOptionsToPackage{hyphens}{url}
%
\documentclass[
]{article}
\usepackage{amsmath,amssymb}
\usepackage{lmodern}
\usepackage{iftex}
\ifPDFTeX
  \usepackage[T1]{fontenc}
  \usepackage[utf8]{inputenc}
  \usepackage{textcomp} % provide euro and other symbols
\else % if luatex or xetex
  \usepackage{unicode-math}
  \defaultfontfeatures{Scale=MatchLowercase}
  \defaultfontfeatures[\rmfamily]{Ligatures=TeX,Scale=1}
\fi
% Use upquote if available, for straight quotes in verbatim environments
\IfFileExists{upquote.sty}{\usepackage{upquote}}{}
\IfFileExists{microtype.sty}{% use microtype if available
  \usepackage[]{microtype}
  \UseMicrotypeSet[protrusion]{basicmath} % disable protrusion for tt fonts
}{}
\makeatletter
\@ifundefined{KOMAClassName}{% if non-KOMA class
  \IfFileExists{parskip.sty}{%
    \usepackage{parskip}
  }{% else
    \setlength{\parindent}{0pt}
    \setlength{\parskip}{6pt plus 2pt minus 1pt}}
}{% if KOMA class
  \KOMAoptions{parskip=half}}
\makeatother
\usepackage{xcolor}
\usepackage[margin=1in]{geometry}
\usepackage{color}
\usepackage{fancyvrb}
\newcommand{\VerbBar}{|}
\newcommand{\VERB}{\Verb[commandchars=\\\{\}]}
\DefineVerbatimEnvironment{Highlighting}{Verbatim}{commandchars=\\\{\}}
% Add ',fontsize=\small' for more characters per line
\usepackage{framed}
\definecolor{shadecolor}{RGB}{248,248,248}
\newenvironment{Shaded}{\begin{snugshade}}{\end{snugshade}}
\newcommand{\AlertTok}[1]{\textcolor[rgb]{0.94,0.16,0.16}{#1}}
\newcommand{\AnnotationTok}[1]{\textcolor[rgb]{0.56,0.35,0.01}{\textbf{\textit{#1}}}}
\newcommand{\AttributeTok}[1]{\textcolor[rgb]{0.77,0.63,0.00}{#1}}
\newcommand{\BaseNTok}[1]{\textcolor[rgb]{0.00,0.00,0.81}{#1}}
\newcommand{\BuiltInTok}[1]{#1}
\newcommand{\CharTok}[1]{\textcolor[rgb]{0.31,0.60,0.02}{#1}}
\newcommand{\CommentTok}[1]{\textcolor[rgb]{0.56,0.35,0.01}{\textit{#1}}}
\newcommand{\CommentVarTok}[1]{\textcolor[rgb]{0.56,0.35,0.01}{\textbf{\textit{#1}}}}
\newcommand{\ConstantTok}[1]{\textcolor[rgb]{0.00,0.00,0.00}{#1}}
\newcommand{\ControlFlowTok}[1]{\textcolor[rgb]{0.13,0.29,0.53}{\textbf{#1}}}
\newcommand{\DataTypeTok}[1]{\textcolor[rgb]{0.13,0.29,0.53}{#1}}
\newcommand{\DecValTok}[1]{\textcolor[rgb]{0.00,0.00,0.81}{#1}}
\newcommand{\DocumentationTok}[1]{\textcolor[rgb]{0.56,0.35,0.01}{\textbf{\textit{#1}}}}
\newcommand{\ErrorTok}[1]{\textcolor[rgb]{0.64,0.00,0.00}{\textbf{#1}}}
\newcommand{\ExtensionTok}[1]{#1}
\newcommand{\FloatTok}[1]{\textcolor[rgb]{0.00,0.00,0.81}{#1}}
\newcommand{\FunctionTok}[1]{\textcolor[rgb]{0.00,0.00,0.00}{#1}}
\newcommand{\ImportTok}[1]{#1}
\newcommand{\InformationTok}[1]{\textcolor[rgb]{0.56,0.35,0.01}{\textbf{\textit{#1}}}}
\newcommand{\KeywordTok}[1]{\textcolor[rgb]{0.13,0.29,0.53}{\textbf{#1}}}
\newcommand{\NormalTok}[1]{#1}
\newcommand{\OperatorTok}[1]{\textcolor[rgb]{0.81,0.36,0.00}{\textbf{#1}}}
\newcommand{\OtherTok}[1]{\textcolor[rgb]{0.56,0.35,0.01}{#1}}
\newcommand{\PreprocessorTok}[1]{\textcolor[rgb]{0.56,0.35,0.01}{\textit{#1}}}
\newcommand{\RegionMarkerTok}[1]{#1}
\newcommand{\SpecialCharTok}[1]{\textcolor[rgb]{0.00,0.00,0.00}{#1}}
\newcommand{\SpecialStringTok}[1]{\textcolor[rgb]{0.31,0.60,0.02}{#1}}
\newcommand{\StringTok}[1]{\textcolor[rgb]{0.31,0.60,0.02}{#1}}
\newcommand{\VariableTok}[1]{\textcolor[rgb]{0.00,0.00,0.00}{#1}}
\newcommand{\VerbatimStringTok}[1]{\textcolor[rgb]{0.31,0.60,0.02}{#1}}
\newcommand{\WarningTok}[1]{\textcolor[rgb]{0.56,0.35,0.01}{\textbf{\textit{#1}}}}
\usepackage{longtable,booktabs,array}
\usepackage{calc} % for calculating minipage widths
% Correct order of tables after \paragraph or \subparagraph
\usepackage{etoolbox}
\makeatletter
\patchcmd\longtable{\par}{\if@noskipsec\mbox{}\fi\par}{}{}
\makeatother
% Allow footnotes in longtable head/foot
\IfFileExists{footnotehyper.sty}{\usepackage{footnotehyper}}{\usepackage{footnote}}
\makesavenoteenv{longtable}
\usepackage{graphicx}
\makeatletter
\def\maxwidth{\ifdim\Gin@nat@width>\linewidth\linewidth\else\Gin@nat@width\fi}
\def\maxheight{\ifdim\Gin@nat@height>\textheight\textheight\else\Gin@nat@height\fi}
\makeatother
% Scale images if necessary, so that they will not overflow the page
% margins by default, and it is still possible to overwrite the defaults
% using explicit options in \includegraphics[width, height, ...]{}
\setkeys{Gin}{width=\maxwidth,height=\maxheight,keepaspectratio}
% Set default figure placement to htbp
\makeatletter
\def\fps@figure{htbp}
\makeatother
\usepackage[normalem]{ulem}
\setlength{\emergencystretch}{3em} % prevent overfull lines
\providecommand{\tightlist}{%
  \setlength{\itemsep}{0pt}\setlength{\parskip}{0pt}}
\setcounter{secnumdepth}{-\maxdimen} % remove section numbering
\ifLuaTeX
  \usepackage{selnolig}  % disable illegal ligatures
\fi
\IfFileExists{bookmark.sty}{\usepackage{bookmark}}{\usepackage{hyperref}}
\IfFileExists{xurl.sty}{\usepackage{xurl}}{} % add URL line breaks if available
\urlstyle{same} % disable monospaced font for URLs
\hypersetup{
  pdftitle={7: Crafting Reports},
  pdfauthor={Environmental Data Analytics \textbar{} John Fay \& Luana Lima \textbar{} Developed by Kateri Salk},
  hidelinks,
  pdfcreator={LaTeX via pandoc}}

\title{7: Crafting Reports}
\author{Environmental Data Analytics \textbar{} John Fay \& Luana Lima
\textbar{} Developed by Kateri Salk}
\date{Spring 2023}

\begin{document}
\maketitle

{
\setcounter{tocdepth}{2}
\tableofcontents
}
\hypertarget{lesson-objectives}{%
\subsection{LESSON OBJECTIVES}\label{lesson-objectives}}

\begin{enumerate}
\def\labelenumi{\arabic{enumi}.}
\tightlist
\item
  Describe the purpose of using R Markdown as a communication and
  workflow tool
\item
  Incorporate Markdown syntax into documents
\item
  Communicate the process and findings of an analysis session in the
  style of a report
\end{enumerate}

\hypertarget{use-of-r-studio-r-markdown-so-far}{%
\subsection{USE OF R STUDIO \& R MARKDOWN SO
FAR\ldots{}}\label{use-of-r-studio-r-markdown-so-far}}

\begin{enumerate}
\def\labelenumi{\arabic{enumi}.}
\tightlist
\item
  Write code
\item
  Document that code
\item
  Generate PDFs of code and its outputs
\item
  Integrate with Git/GitHub for version control
\end{enumerate}

\hypertarget{basic-r-markdown-document-structure}{%
\subsection{BASIC R MARKDOWN DOCUMENT
STRUCTURE}\label{basic-r-markdown-document-structure}}

\begin{enumerate}
\def\labelenumi{\arabic{enumi}.}
\tightlist
\item
  \textbf{YAML Header} surrounded by --- on top and bottom

  \begin{itemize}
  \tightlist
  \item
    YAML templates include options for html, pdf, word, markdown, and
    interactive
  \item
    More information on formatting the YAML header can be found in the
    cheat sheet
  \end{itemize}
\item
  \textbf{R Code Chunks} surrounded by
  ``\texttt{on\ top\ and\ bottom\ \ \ \ \ +\ Create\ using}Cmd/Ctrl\texttt{+}Alt\texttt{+}I`

  \begin{itemize}
  \tightlist
  \item
    Can be named \{r name\} to facilitate navigation and autoreferencing
  \item
    Chunk options allow for flexibility when the code runs and when the
    document is knitted
  \end{itemize}
\item
  \textbf{Text} with formatting options for readability in knitted
  document
\end{enumerate}

\hypertarget{resources}{%
\subsection{RESOURCES}\label{resources}}

Handy cheat sheets for R markdown can be found:
\href{https://rstudio.com/wp-content/uploads/2015/03/rmarkdown-reference.pdf}{here},
and
\href{https://raw.githubusercontent.com/rstudio/cheatsheets/master/rmarkdown-2.0.pdf}{here}.

There's also a quick reference available via the
\texttt{Help}→\texttt{Markdown\ Quick\ Reference} menu.

Lastly, this \href{https://rmarkdown.rstudio.com}{website} give a great
\& thorough overview.

\hypertarget{the-knitting-process}{%
\subsection{THE KNITTING PROCESS}\label{the-knitting-process}}

\begin{itemize}
\item
  The knitting sequence\\
  \includegraphics{./img/rmarkdownflow.png}
\item
  Knitting commands in code chunks:

  \begin{itemize}
  \tightlist
  \item
    \texttt{include\ =\ FALSE} - code is run, but neither code nor
    results appear in knitted file
  \item
    \texttt{echo\ =\ FALSE} - code not included in knitted file, but
    results are
  \item
    \texttt{eval\ =\ FALSE} - code is not run in the knitted file
  \item
    \texttt{message\ =\ FALSE} - messages do not appear in knitted file
  \item
    \texttt{warning\ =\ FALSE} - warnings do not appear\ldots{}
  \item
    \texttt{fig.cap\ =\ "..."} - adds a caption to graphical results
  \end{itemize}
\end{itemize}

\hypertarget{what-else-can-r-markdown-do}{%
\subsection{WHAT ELSE CAN R MARKDOWN
DO?}\label{what-else-can-r-markdown-do}}

See: \url{https://rmarkdown.rstudio.com} and class recording.

\begin{itemize}
\tightlist
\item
  Languages other than R\ldots{}
\item
  Various outputs\ldots{}
\end{itemize}

\begin{center}\rule{0.5\linewidth}{0.5pt}\end{center}

\hypertarget{why-r-markdown}{%
\subsection{WHY R MARKDOWN?}\label{why-r-markdown}}

\begin{quote}
Fill in our discussion below with bullet points. Use italics and bold
for emphasis (hint: use the cheat sheets or \texttt{Help}
→\texttt{Markdown\ Quick\ Reference} to figure out how to make bold and
italic text).
\end{quote}

\begin{itemize}
\tightlist
\item
  R Markdown is limited to \sout{one language} \textbf{many languages}
\item
  (Add more bullets here)
\end{itemize}

\hypertarget{text-editing-challenge}{%
\subsection{TEXT EDITING CHALLENGE}\label{text-editing-challenge}}

Create a table below that details the example datasets we have been
using in class. The first column should contain the names of the
datasets and the second column should include some relevant information
about the datasets. (Hint: use the cheat sheets to figure out how to
make a table in Rmd)

\begin{longtable}[]{@{}ll@{}}
\toprule()
File name & Description \\
\midrule()
\endhead
NWIS\_SiteInfo\_NE\_RAW.csv & NWIS Site Information \\
NTL-LTER\_Lake\_Carbon\_Raw.csv & NTL LTER Lake: Carbon data \\
\bottomrule()
\end{longtable}

\newpage

\hypertarget{r-chunk-editing-challenge}{%
\subsection{R CHUNK EDITING CHALLENGE}\label{r-chunk-editing-challenge}}

\hypertarget{installing-packages}{%
\subsubsection{Installing packages}\label{installing-packages}}

Create an R chunk below that installs the package \texttt{knitr}.
Instead of commenting out the code, customize the chunk options such
that the code is not evaluated (i.e., not run).

\begin{Shaded}
\begin{Highlighting}[]
\FunctionTok{install}\NormalTok{(}\StringTok{\textquotesingle{}knitr\textquotesingle{}}\NormalTok{)}
\end{Highlighting}
\end{Shaded}

\hypertarget{setup}{%
\subsubsection{Setup}\label{setup}}

Create an R chunk below called ``setup'' that loads the packages
\texttt{tidyverse}, \texttt{lubridate}, \texttt{here}, and
\texttt{knitr} and sets a ggplot theme. Remember that you need to
disable R throwing a message, which contains a check mark that cannot be
knitted.

\begin{Shaded}
\begin{Highlighting}[]
\CommentTok{\#Load packages}
\FunctionTok{library}\NormalTok{(tidyverse)}
\FunctionTok{library}\NormalTok{(lubridate)}
\FunctionTok{library}\NormalTok{(knitr)}
\FunctionTok{library}\NormalTok{(here)}

\CommentTok{\#Set theme}
\NormalTok{my\_theme }\OtherTok{\textless{}{-}} \FunctionTok{theme\_light}\NormalTok{() }\SpecialCharTok{+} 
  \FunctionTok{theme}\NormalTok{(}\AttributeTok{axis.line =} \FunctionTok{element\_line}\NormalTok{(}\AttributeTok{color =} \StringTok{"lightblue"}\NormalTok{))}

\FunctionTok{theme\_set}\NormalTok{(my\_theme)}
\end{Highlighting}
\end{Shaded}

In a new code chunk named ``load data'', load the
NTL-LTER\_Lake\_Nutrients\_Raw dataset, convert dates, display the head
of the dataset, and set the date column to a date format.Customize the
chunk options such that the code is run but is not displayed in the
knitted document. The output, however, should be displayed.

\begin{verbatim}
##   lakeid  lakename year4 daynum sampledate depth_id depth tn_ug tp_ug nh34 no23
## 1      L Paul Lake  1991    140 1991-05-20        1  0.00   538    25   NA   NA
## 2      L Paul Lake  1991    140 1991-05-20        2  0.85   285    14   NA   NA
## 3      L Paul Lake  1991    140 1991-05-20        3  1.75   399    14   NA   NA
## 4      L Paul Lake  1991    140 1991-05-20        4  3.00   453    14   NA   NA
## 5      L Paul Lake  1991    140 1991-05-20        5  4.00   363    13   NA   NA
## 6      L Paul Lake  1991    140 1991-05-20        6  6.00   583    37   NA   NA
##   po4 comments
## 1  NA         
## 2  NA         
## 3  NA         
## 4  NA         
## 5  NA         
## 6  NA
\end{verbatim}

\hypertarget{data-exploration-wrangling-and-visualization}{%
\subsubsection{Data Exploration, Wrangling, and
Visualization}\label{data-exploration-wrangling-and-visualization}}

Create an R chunk below to create a processed dataset do the following
operations:

\begin{itemize}
\tightlist
\item
  Include all columns except lakeid, depth\_id, and comments
\item
  Include only surface samples (depth = 0 m)
\item
  Drop rows with missing data
\end{itemize}

\begin{Shaded}
\begin{Highlighting}[]
\CommentTok{\#Wrangle the data}
\NormalTok{ntl.lter.processed }\OtherTok{\textless{}{-}}\NormalTok{ ntl.lter.raw }\SpecialCharTok{\%\textgreater{}\%} 
  \FunctionTok{select}\NormalTok{(}\SpecialCharTok{{-}}\FunctionTok{c}\NormalTok{(lakeid,depth\_id,comments)) }\SpecialCharTok{\%\textgreater{}\%} \CommentTok{\#Subset columns}
  \FunctionTok{filter}\NormalTok{(depth }\SpecialCharTok{==} \DecValTok{0}\NormalTok{) }\SpecialCharTok{\%\textgreater{}\%}                   \CommentTok{\#Select only surface samples}
  \FunctionTok{drop\_na}\NormalTok{()                                }\CommentTok{\#Drop rows with missing data}
\end{Highlighting}
\end{Shaded}

Create a second R chunk to create a summary dataset with the mean,
minimum, maximum, and standard deviation of total nitrogen
concentrations for each lake. Create a second summary dataset that is
identical except that it evaluates total phosphorus. Customize the chunk
options such that the code is run but not displayed in the final
document.

Create a third R chunk that uses the function \texttt{kable} in the
knitr package to display two tables: one for the summary dataframe for
total N and one for the summary dataframe of total P. Use the
\texttt{caption\ =\ "\ "} code within that function to title your
tables. Customize the chunk options such that the final table is
displayed but not the code used to generate the table.

\begin{longtable}[]{@{}lrrrr@{}}
\caption{Surface Samples: Total Nitrogen}\tabularnewline
\toprule()
lakename & mean\_tn\_ug & min\_tn\_ug & max\_tn\_ug & sd\_tn\_ug \\
\midrule()
\endfirsthead
\toprule()
lakename & mean\_tn\_ug & min\_tn\_ug & max\_tn\_ug & sd\_tn\_ug \\
\midrule()
\endhead
Central Long Lake & 690.0469 & 343.020 & 953.063 & 209.09341 \\
Crampton Lake & 362.6813 & 353.380 & 376.304 & 12.05748 \\
East Long Lake & 810.7834 & 380.620 & 2608.956 & 335.41457 \\
Hummingbird Lake & 1036.6695 & 779.053 & 1221.960 & 204.36889 \\
Paul Lake & 368.7564 & 45.670 & 628.625 & 106.34741 \\
Peter Lake & 561.8752 & 219.720 & 2048.151 & 305.64909 \\
Tuesday Lake & 423.5605 & 237.363 & 554.418 & 78.84522 \\
West Long Lake & 762.6017 & 303.170 & 2870.302 & 402.95992 \\
\bottomrule()
\end{longtable}

\begin{longtable}[]{@{}lrrrr@{}}
\caption{Surface Samples: Total Phosphorous}\tabularnewline
\toprule()
lakename & mean\_tp\_ug & min\_tp\_ug & max\_tp\_ug & sd\_tp\_ug \\
\midrule()
\endfirsthead
\toprule()
lakename & mean\_tp\_ug & min\_tp\_ug & max\_tp\_ug & sd\_tp\_ug \\
\midrule()
\endhead
Central Long Lake & 21.70981 & 8.190 & 37.270 & 7.076388 \\
Crampton Lake & 11.16033 & 5.803 & 15.555 & 4.946759 \\
East Long Lake & 29.28984 & 8.000 & 101.050 & 17.375710 \\
Hummingbird Lake & 36.21925 & 32.765 & 42.119 & 4.146717 \\
Paul Lake & 10.45606 & 1.222 & 36.070 & 4.805142 \\
Peter Lake & 18.39153 & 0.000 & 64.383 & 10.976205 \\
Tuesday Lake & 11.71853 & 6.325 & 18.663 & 3.044289 \\
West Long Lake & 19.82981 & 2.690 & 63.243 & 10.541276 \\
\bottomrule()
\end{longtable}

Create a fourth and fifth R chunk that generates two plots (one in each
chunk): one for total N over time with different colors for each lake,
and one with the same setup but for total P. Decide which geom option
will be appropriate for your purpose, and select a color palette that is
visually pleasing and accessible. Customize the chunk options such that
the final figures are displayed but not the code used to generate the
figures. In addition, customize the chunk options such that the figures
are aligned on the left side of the page. Lastly, add a fig.cap chunk
option to add a caption (title) to your plot that will display
underneath the figure.

\begin{figure}

\includegraphics{07_Part1_CraftingReports_Complete_files/figure-latex/plot.nitrogen-1} \hfill{}

\caption{Total Nitrogen}\label{fig:plot.nitrogen}
\end{figure}

\begin{figure}

\includegraphics{07_Part1_CraftingReports_Complete_files/figure-latex/plot.phosphorus-1} \hfill{}

\caption{Total Phosphorous}\label{fig:plot.phosphorus}
\end{figure}

\hypertarget{communicating-results}{%
\subsubsection{Communicating results}\label{communicating-results}}

Write a paragraph describing your findings from the R coding challenge
above. This should be geared toward an educated audience but one that is
not necessarily familiar with the dataset. Then insert a horizontal rule
below the paragraph. Below the horizontal rule, write another paragraph
describing the next steps you might take in analyzing this dataset. What
questions might you be able to answer, and what analyses would you
conduct to answer those questions?

\hypertarget{knit-your-pdf}{%
\subsection{KNIT YOUR PDF}\label{knit-your-pdf}}

When you have completed the above steps, try knitting your PDF to see if
all of the formatting options you specified turned out as planned. This
may take some troubleshooting.

\hypertarget{other-r-markdown-customization-options}{%
\subsection{OTHER R MARKDOWN CUSTOMIZATION
OPTIONS}\label{other-r-markdown-customization-options}}

We have covered the basics in class today, but R Markdown offers many
customization options. A word of caution: customizing templates will
often require more interaction with LaTeX and installations on your
computer, so be ready to troubleshoot issues.

Customization options for pdf output include:

\begin{itemize}
\tightlist
\item
  Table of contents
\item
  Number sections
\item
  Control default size of figures
\item
  Citations
\item
  Template (more info
  \href{http://jianghao.wang/post/2017-12-08-rmarkdown-templates/}{here})
\end{itemize}

pdf\_document:\\
toc: true\\
number\_sections: true\\
fig\_height: 3\\
fig\_width: 4\\
citation\_package: natbib\\
template:

\end{document}
